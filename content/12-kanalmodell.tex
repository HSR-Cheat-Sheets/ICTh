%! Author = mariuszindel
%! Date = 25.01.21

\section{Kanalmodell}





\section{Komprimierung}


\subsection{Huffman}
Berechnung der mittleren Code Wortlänge =\\ \colorbox{lightlightgrey}{$P(x)*L(x)$} Auftrittswahrscheinlichkeit * Bitlänge des Zeichens


\subsection{Codierung}


\subsubsection{DNA}
Anzahl gleiche Buchstaben + Buchstaben $\rightarrow$ aaaxxc = 3a2xc


\subsubsection{Bit-Folgen}
Bestimmen mit welchem Wert(0 oder 1) man beginnt und dann nur die Anzahl der Folge aufschreiben\\
000111010 $\rightarrow$ Beginnen mit 0 = 3 3 1 1 1\\Da die 3 mit 2 Bits dargestellt werden kann, muss man die Zeichenfolge mit 2 Bits darstellen:(Beginnend mit dem ausgewähltem Wert als 1 Bit)\\
0 11 11 01 01 01









\section{Kanalcodierung}
Alle folgenden \textcolor{red}{$p(y_i|x_i)$} Werte beziehen sich auf die Kanalmatrix!\\\\
\begin{minipage}[t]{0.15\textwidth}
    Ein \textcolor{red}{nicht} gestörter Kanal hat als Kanalmatrix die Einheitsmatrix:\\
    $P(Y|X) = \begin{bmatrix}
                  1 & 0 & 0\\
                  0 & 1 & 0\\
                  0 & 0 & 1
    \end{bmatrix}$
\end{minipage}
\hfill
\begin{minipage}[t]{0.15\textwidth}
    Ein \textcolor{red}{vollständig} gestörter Kanal hat die Wahrscheinlichkeiten gleichverteilt:\\
    $P(Y|X) = \begin{bmatrix}
                  0.3 & 0.3 & 0.3\\
                  0.3 & 0.3 & 0.3\\
                  0.3 & 0.3 & 0.3
    \end{bmatrix}$
\end{minipage}


\subsection{Entropie H(X) = Eingang oder H(Y) = Ausgang}
\textcolor{red}{$H(X)$} wird ganz normal berechnet\\\\
\textcolor{red}{$H(Y)$} \\
\colorbox{lightlightgrey}{$p(y_1) = p(x_1)*p(y_1|x_1) + p(x_2)*p(y_1|x_2) + p(x_3)*p(y_1|x_3)$}\\
\colorbox{lightlightgrey}{$p(y_2) = p(x_1)*p(y_2|x_1) + p(x_2)*p(y_2|x_2) + p(x_3)*p(y_2|x_3)$}\\
\colorbox{lightlightgrey}{$p(y_3) = p(x_1)*p(y_3|x_1) + p(x_2)*p(y_3|x_2) + p(x_3)*p(y_3|x_3)$}\\
Mit diesen berechneten Wahrscheinlichkeiten für ein Zeichen, kann die Entropie ganz normal berechnet werden


\subsection{Irrelevanz $H(Y|X)$}


\colorbox{lightlightgrey}{$H(Y|X) = -p(x_i)*p(y_k|x_i) * log_2(p(y_k|x_i))$}\\
Diese Formel wird für \textcolor{red}{jede} Zelle($p(y_k|x_i)$) in der \textcolor{blue}{Kanalmatrix} berechnet und voneinander subtrahiert
\subsection{Transinformation bit/Zeichen}
\colorbox{lightlightgrey}{$T = H(Y) - H(Y|X)$ bit/Zeichen}



\subsection{Maximale Symbolrate bit/s}
$R_{max} = T *$ Übertragungsrate
\subsection{verlustfreie, rauschbehafteter Kanal}
$P(Y|X) = \begin{bmatrix}
              1 & 0 & 0\\
              0 & 0.2 & 0.8
\end{bmatrix}$



\subsection{Kanalmatrix aus Wahrscheinlichkeiten bestimmen}
\begin{minipage}[t]{0.15\textwidth}
    $x_1 = 0.4, x_2 = 0.6$\\
    $y_1 = 0.8, y_2 = 0.2$\\
    \colorbox{lightlightgrey}{$P(Y) = P(Y|X) * P(X)$}\\
    \colorbox{lightlightgrey}{$\begin{bmatrix}
                                   0.8\\
                                   0.2
    \end{bmatrix}$ =
    $\begin{bmatrix}
         a & b\\
         b & a
    \end{bmatrix}$ *
    $\begin{bmatrix}
         0.4\\
         0.6
    \end{bmatrix}$}
\end{minipage}
\hfill
\begin{minipage}[t]{0.15\textwidth}
    $0.8 = 0.4a + 0.6b$\\
    $0.2 = 0.4b + 0.6a$\\
\end{minipage}









