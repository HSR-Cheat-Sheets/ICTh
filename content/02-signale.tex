%! Author = mariuszindel
%! Date = 25.01.21

\section{Signale}


\subsection{Periodendauer}
Die Periodendauer hat den Formelbuchstaben T und die Einheit Sekunde s.\\
$T[s] = \frac{1}{f}$

\subsection{Frequenz}
Die Frequenz hat den Formelbuchstaben f und die Einheit Hertz Hz.\\
$f[Hz] = \frac{1}{T}$



\subsection{Dauer und Bandbreite von Einzelpulsen}
\subsubsection{Vorgehen Amplitudendichte}
\textbf{Fragestellung:} Wie kann S(0), d.h. die Amplitudendichte bei der Frequenz f=0 Hz, einfach aus dem Verlauf des Pulses berechnet werden? Geben Sie die Formel für S(0), sowie den numerischen Wert in [V/Hz] an.\\
\\
\textbf{Lösung:} $S(0)=\int_{-\infty }^{\infty}s(t)dt=\int_{-T}^{T}s(t)dt=AT$ \\
\\
d.h. die Gesamtfläche unter der Dreiecksfunktion s(t).\\
\\
\textbf{Hinweis:} \\
Bei Recktecksignalen wäre es $2*AT$\\
Bei ms führt es zu mV/Hz

\subsubsection{Energie berechnen}
$E=\int_{-\infty}^{\infty}\frac{s^2(t)}{R}dt=\frac{1}{R}\int_{-\infty}^{\infty}s^2(t)dt=\frac{3}{4}*\frac{A^2T}{R}$
\\
\\
Achtung: 3/4 und T ist modular \\
\\
\textbf{Hinweis:} Resultat wenn ms dann mWs (Beispiel: 3/4mWs = 0.75mJ)

\subsubsection{Dauer eines Pulses berechnen}
\textbf{Formel:} $E=\frac{A^2\tau}{R}=\frac{3}{4}*\frac{A^2T}{R}$ \\
\textbf{nach $\tau$ auflösen $\tau$ in ms oder s}\\
\\
Achtung: 3/4 und T ist modular

\subsubsection{Bandbreite}
\textbf{Formel:} $E=\frac{|S(0)|^2*B}{R}=\frac{A^2T^2B}{R}=\frac{3}{4}*\frac{A^2T}{R}$\\ $B=\frac{3}{4}*\frac{1}{T}$ und damit $B=\frac{3}{4}kHz=0.75kHz$\\
\\
Achtung: 3/4 und T ist modular

\subsubsection{Zeit-Bandbreitenprodukt}
Wie gross ist das Zeit-Bandbreitenprodukt B\tau ?\\
\\
\textbf{Formel:} $B\tau=\frac{3}{4}*\frac{1}{T}*\frac{3}{4}*T$
\\
\\
Achtung: 3/4 und T ist modular
\\
\textbf{Hinweis:} kHz und ms lösen sich auf = keine Masseinheit


% \subsection{Frequenzspektrum}


% \subsection{Analyse der Spektrallinien}


% \subsection{Signalleistung im Frequenz- und Zeitbereich}


% \subsection{Logarithmische Spektrumsdarstellung}


% \subsection{Zeitverschobene periodische Signale}