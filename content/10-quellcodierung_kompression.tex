%! Author = mariuszindel
%! Date = 25.01.21

\section{Kompression}


\subsection{Huffman}
Berechnung der mittleren Code Wortlänge =\\ \colorbox{lightlightgrey}{$P(x)*L(x)$} Auftrittswahrscheinlichkeit * Bitlänge des Zeichens


\subsection{Codierung}


\subsubsection{DNA}
Anzahl gleiche Buchstaben + Buchstaben $\rightarrow$ aaaxxc = 3a2xc


\subsubsection{Bit-Folgen}
Bestimmen mit welchem Wert(0 oder 1) man beginnt und dann nur die Anzahl der Folge aufschreiben\\
000111010 $\rightarrow$ Beginnen mit 0 = 3 3 1 1 1\\Da die 3 mit 2 Bits dargestellt werden kann, muss man die Zeichenfolge mit 2 Bits darstellen:(Beginnend mit dem ausgewähltem Wert als 1 Bit)\\
0 11 11 01 01 01




\vfill
$$
\columnbreak